\subsubsection{base\_sum}

BaseSumGate is used to constrain the input to be composed of limbs which are arranged in little-endian. There are two kinds of constraints:

For each limb, limb is in range [0, base):
\[\sum_{i=0}^{base}(limb_i - i) = 0\]

Input is composed of limbs:
\[input = \sum_{i=0}^{n-1} limb_{n-1-i} * base^i\]

The structure of gate is shown in figure \ref{fig:base-sum}.
\begin{figure}[!ht]
    \centering
    \includegraphics[width=0.8\textwidth]{gates/base_sum.jpeg}
    \caption{BaseSumGate}
    \label{fig:base-sum}
\end{figure}

There's 1 constraint for sum check and 8 constraints for limbs' range check. Degree of the gate is $2^{limb\_size}$ happens when limbs' range check.